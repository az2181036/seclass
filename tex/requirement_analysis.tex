\documentclass[10pt,journal,compsoc,fleqn]{IEEEtran}
\usepackage{graphicx}
\usepackage{amsmath}
\usepackage{amsfonts}
\usepackage{epstopdf}
\usepackage{CJKutf8}
\usepackage{algorithm, algorithmic}
\newtheorem{thm}{{Theorem}}[section]
\newtheorem{defn}{Definition}[section]
\newtheorem{lem}{Lemma}[section]
\newtheorem{rmk}{Remark}[section]
\newenvironment{proof}{\noindent {\bf Proof: }}\\


\hyphenation{op-tical net-works semi-conduc-tor}


\begin{document}
\begin{CJK}{UTF8}{song}
\title{Requirement Analysis}
\author{龙威帆}
\maketitle

\section{UI}
一个输入出题数的textedit.\\
一个提交按钮 (初始可点击).\\
五个算式 (gridlayout):

四个数字 (答案可编辑,其余不可编辑);

两个算数符 + “=” (lable) 不可编辑;

前两个数 可能存在括号 (lable) 不可编辑.\\
上一页,下一页 button(初始不可点击).\\
显示答案,检查对错 button(初始不可点击).\\


\section{响应}
Commit button: 出题数 输入 为 字符、非正整数、大于1000: warning\\
Last page: 刷新, 显示上5道题\\
Next Page: 刷新,显示下5(可能小于5)道题\\
View answers: 显示各题答案\\
Check results: 检测各题答案输入,绿色正确,红色错误

\section{刷新}
Clear 网格布局里所有的text 和 lable\\
重置颜色

\section{显示}
判断题目数是否还有5个,加if判断,超过则跳过。\\
获取类中数据,把数据提取到各个等式。

	lst:

		\quad Flag: 是否有括号  index[-2]

		\quad Num\_1,2,3: 三个操作数 index[0,1,2]

		\quad Sum\_123: 答案 index[-1]

		\quad Op1,2: 算数符 index[3,4]

\end{CJK}
\end{document}